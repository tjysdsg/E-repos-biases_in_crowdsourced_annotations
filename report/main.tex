% This must be in the first 5 lines to tell arXiv to use pdfLaTeX, which is strongly recommended.
\pdfoutput=1
% In particular, the hyperref package requires pdfLaTeX in order to break URLs across lines.

\documentclass[11pt]{article}

% Remove the "review" option to generate the final version.
% \usepackage[review]{acl}
\usepackage{acl}

\usepackage{times}
\usepackage{latexsym}

% For proper rendering and hyphenation of words containing Latin characters (including in bib files)
\usepackage[T1]{fontenc}
% For Vietnamese characters
% \usepackage[T5]{fontenc}
% See https://www.latex-project.org/help/documentation/encguide.pdf for other character sets

\usepackage[utf8]{inputenc}
\usepackage{microtype}
\usepackage{amsmath}
\usepackage{threeparttable}

% Set the left margin to zero for the quote environment
\AtBeginEnvironment{quote}{\setlength{\leftmargini}{4pt}}


\title{11830 Report: Biases in Crowdsourced Annotations}

\author{
    Jiyang Tang \\
    \texttt{jiyangta@andrew.cmu.edu}
}

\begin{document}
    \maketitle
    % \begin{abstract}
    % \end{abstract}


    \section{Introduction}

    The Stanford Natural Language Inference (SNLI) corpus\footnote{\url{https://nlp.stanford.edu/projects/snli/}} is a
    large crowdsourced natural language inference dataset~\cite{snli}.
    For each premise sentence, annotators were asked to write a hypothesis that is either a contradiction,
    a neutral statement, or entailment of the premise.

    The data was collected from Amazon Mechanical Turk (MTurk) crowdsourcing service.
    Previous studies have shown that MTurk crowd-workers tend to have lower income, higher education levels, and
    lower average ages~\cite{mturks_demography}.
    We believe that biases are propagated from the dataset authors to the premise text, and from the
    crowd-workers to the hypothesis text.
    Therefore, we will analyze the biases in the data by performing word association tests in this report.


    \section{Method}

    \subsection{Pointwise Mutual Information}

    Pointwise Mutual Information (PMI) is used to measure how much word $w_i$ is associated with the word
    $w_j$~\cite{pmi,speech_and_nlp_book}.
    PMI is calculated as follows:

    \[
        \text{PMI}(w_i, w_j) = log_2\frac{N\cdot c(w_i, w_j)}{c(w_i)c(w_j)}
    \]
    where $N$ is the total number of sentences in the corpus,
    $c(w_i,w_j)$ is number of times $w_i$ and $w_j$ co-occur in a sentence,
    $c(w_i)$ is the number of times $w_i$ occurs in the corpus,
    and $c(w_j)$ is the number of times $w_j$ occurs in the corpus.

    Note that if a pair of words $w_i$ and $w_j$ occurs multiple times in the same sentence, $c(w_i,w_j)$ is counted
    as $1$.

    PMI ranges from negative infinity to positive infinity.
    Large PMI suggests high word association.
    Negative PMI implies two words co-occur less often than by chance and is unreliable in
    practice~\cite{speech_and_nlp_book}.


    \section{Experiments}

    \subsection{Data Preprocessing}

    As mentioned before, each data sample contains a premise and a hypothesis.
    Note that multiple hypotheses might be generated from the same premise, therefore duplicated sentences are
    removed.
    All words are converted into their lowercase form and stop words are removed.
    Then we use \texttt{spaCy}~\cite{spacy} \texttt{en\_core\_web\_sm} model to tokenize raw strings into
    lists of words and remove all punctuations.

    Note that words that occur less than $10$ times in the corpus are removed.

    \subsection{Unigram PMI}

    We individually perform unigram PMI analysis on the premise text and the hypothesis text.
    For each analysis experiment, we focus on the most associated words with a set of
    identity labels~\cite{identity_labels}.

    \begin{table*}
        \begin{threeparttable}
            \small
            \centering
            \begin{tabular}{l|c|c}
                \hline
                Identity & Premise & Hypothesis \\
                \hline
                women &
                saris,
                headscarves,
                bikinis,
                headdresses,
                coverings
                &
                burkas,
                husbands,
                saris,
                kimonos,
                bikinis
                \\\hline

                men &
                turbans,
                tuxedos,
                ladders,
                jumpsuits,
                wetsuits,
                &
                turbans,
                rickshaws,
                wives,
                cigars,
                tuxedos,
                \\\hline

                africans &
                tribe,
                hearts,
                tap,
                huts
                &
                armed,
                source,
                die,
                cloths,
                tribal
                \\\hline

                caucasian &
                lockers,
                handsome,
                explains,
                contemplates,
                straddling
                &
                slender,
                fleece,
                non,
                zip,
                festive
                \\\hline

                muslims\tnote{1} &
                channel,
                news,
                sponsored,
                celebrate,
                speech
                &
                christians,
                terrorists,
                celebrate,
                opening,
                phones
                \\\hline

                christians\tnote{1} &
                praising,
                lord,
                crazy,
                fun,
                woods,
                &
                muslims,
                gospel,
                impressed,
                pork,
                villagers
                \\\hline

                gay &
                pride,
                marriage,
                attendees,
                protester,
                participants,
                &
                pride,
                rights,
                marriage,
                experimenting,
                abraham
                \\\hline

                straight &
                razor,
                ahead,
                stony,
                sketch,
                crack
                &
                razor,
                tambourine,
                ahead,
                stared,
                lanes
                \\\hline

                israeli &
                desolate,
                nuts,
                pirates,
                cigarettes,
                u.s.
                &
                problems,
                eastern,
                cashier,
                cigarettes,
                counter
                \\\hline

                american &
                footballer,
                african,
                native,
                patriotic,
                south
                &
                idol,
                native,
                african,
                latin,
                drapes
                \\\hline

                indian &
                sari,
                headdresses,
                saris,
                ritual,
                chief
                &
                southeast,
                style,
                boot,
                muffins,
                descent
                \\\hline

            \end{tabular}

            \begin{tablenotes}
                \item[1] Words that don't have associated words that occur more than 10 times in premise text.
                The threshold is ignored for these words.
            \end{tablenotes}
        \end{threeparttable}

        \caption{Top associated unigrams with identity labels}
        \label{tab:unigram_pmi}
    \end{table*}

    \subsection{Bigram PMI}

    We extend the list of identity labels by combining existing ones into bigrams.
    We also extract all bigrams from the text and add them to the vocabulary for PMI calculation.
    Note that we skip the calculation if two bigrams share a unigram since we are not interested in the association
    between ``he'' and ``he doesn't''.


    \section{Results and Discussion}

    \subsection{Unigram PMI}

    Table~\ref{tab:unigram_pmi} lists the top associated unigrams with some of the identity labels
    in the premises and in the hypotheses.
    We can easily spot some alarmingly biased word associations.
    For example, \texttt{muslims} is highly associated with \texttt{terrorists}, \texttt{africans} co-occur with
    \texttt{armed} and \texttt{die} frequently, while \texttt{caucasian} is often associated with \texttt{handsome}.
    There are also many implicit biases or stereotypes in the data.
    \texttt{women} is most associated with words related to fashion, while \texttt{men} with words about
    tools and work clothes.

    Meanwhile, we can spot more heavily biased word associations in the hypothesis data compared to the premises.
    For example, associations between \texttt{muslims} and \texttt{terrorists} and between
    \texttt{women} and \texttt{gossip} are only present in the hypotheses.

    However, there are yet some cases where the bias is more prevalent on the premises.
    For example, \texttt{israeli} is most associated with \texttt{desolate} and \texttt{pirates} in premises,
    compared to \texttt{problems}, \texttt{eastern}, \texttt{cashier} and so on in hypotheses.

    We also find it interesting that \texttt{south} is among the top 5 most associated words with \texttt{american}
    while \texttt{north} is not found even in the top 20s, as if people think \texttt{american}'s are from North
    America by default.

    \subsection{Bigram PMI}

    \begin{table*}
        \small
        \centering
        \begin{tabular}{l|c|c}
            \hline
            Identity & Premise & Hypothesis \\
            \hline

            old men &
            \begin{tabular}{@{}c@{}}
                park benches,
                stock,
                rock concert, \\
                swim caps,
                concrete bench
            \end{tabular}
            &
            \begin{tabular}{@{}c@{}}
                reminisce,
                straw hats,
                inappropriate, \\
                raincoats,
                grumpy
            \end{tabular}
            \\\hline

            old women &
            \begin{tabular}{@{}c@{}}
                selling vegetables,
                wearing hat, \\
                cases,
                walker,
                fresh produce
            \end{tabular}
            &
            \begin{tabular}{@{}c@{}}
                intense game,
                vases,
                earring, \\
                carring,
                doorstep
            \end{tabular}
            \\\hline

            black people &
            \begin{tabular}{@{}c@{}}
                scarf walks,
                couches,
                looking forward, \\
                boston,
                dilapidated
            \end{tabular}
            &
            \begin{tabular}{@{}c@{}}
                corvette,
                wedding ceremony,
                rioting, \\
                felt,
                robbed
            \end{tabular}
            \\\hline

            black male &
            \begin{tabular}{@{}c@{}}
                puma,
                free throw,
                grinder, \\
                shirt hanging,
                waffle
            \end{tabular}
            &
            \begin{tabular}{@{}c@{}}
                green t-shirt,
                notebooks,
                backyard pool, \\
                building looking,
                unfinished building
            \end{tabular}
            \\\hline

            asian teenagers &
            NA
            &
            \begin{tabular}{@{}c@{}}
                red chairs,
                electronic devices,
                carrying flags, \\
                atm,
                piercings
            \end{tabular}
            \\\hline

        \end{tabular}

        \caption{Top associated unigrams or bigrams with bigram identity labels}
        \label{tab:bigram2unigram_pmi}
    \end{table*}

    Table~\ref{tab:bigram2unigram_pmi} lists the top associated unigrams or bigrams with bigram identity labels.
    This result corresponds to our previous finding that hypothesis text contains more biases and stereotypes.
    For example, \texttt{old men} and \texttt{grumpy}, \texttt{black people} and \texttt{rioting}, and
    \texttt{asian teenagers} and \texttt{electronic devices}.

    \begin{table*}
        \small
        \centering
        \begin{tabular}{l|c|c}
            \hline
            Identity & Premise & Hypothesis \\
            \hline

            indian &
            sari,
            headdresses,
            saris,
            ritual,
            chief
            &
            southeast asian,
            southeast,
            wearing traditional,
            works hard,
            asian descent
            \\\hline

            africans &
            green vegetables,
            tribe,
            hearts,
            tap,
            huts
            &
            shooting guns,
            taking shelter,
            armed,
            source,
            getting water
            \\\hline

        \end{tabular}
        \caption{Top associated unigrams or bigrams with unigram identity labels}
        \label{tab:bigram2bigram_pmi}
    \end{table*}

    Table~\ref{tab:bigram2bigram_pmi} shows the top associated unigrams or bigrams with unigram identity labels.
    We can spot new stereotypical phrases not seen in previous experiments.
    For example, \texttt{indian} and \texttt{works hard}, and \texttt{africans} and \texttt{shooting guns}.

    \subsection{Qualitative Analysis}

    In this section, we will present some examples that contain biases or stereotypes.

    In the example below, the annotator somehow chooses Africans shooting guns as a
    contradictory event of the premise.
    In addition, they use ``shooting guns'' instead of ``hunting''.
    \begin{quote}
        Premise: Africans in tribe clothes, walking pass a green. \\
        Hypothesis: Africans are shooting guns at a bear. \\
        Label: contradiction
    \end{quote}

    In this example, the hypothesis also contains stereotypes against African people.
    \begin{quote}
        Premise: Africans working in a mine digging. \\
        Hypothesis: Africans are being forced to mine for diamonds. \\
        Label: neutral
    \end{quote}

    In the next example, we believe that the annotator is misled by the premise.
    Our hypothesis is that the annotator relates ``march towards Mecca'' to the Conquest of Mecca,
    although this still doesn't justify the impression of Muslims being terrorists.
    \begin{quote}
        Premise: Several Muslim worshipers march towards Mecca. \\
        Hypothesis: The Muslims are terrorists. \\
        Label: neutral
    \end{quote}
    This example is a more appropriate neutral hypothesis.
    \begin{quote}
        Premise: Muslim women talking in a marketplace. \\
        Hypothesis: The Muslims are talking on their phones. \\
        Label: neutral
    \end{quote}

    Below is an example of stereotypes against Israeli people.
    \begin{quote}
        Premise: Five women wearing uniforms and carrying guns are spending time together while a man in a black robe talks on a cellphone in the background. \\
        Hypothesis: The women are Israeli soldiers.\\
        Label: neutral
    \end{quote}

    And in other cases the hypothesis is fine.
    \begin{quote}
        Premise: Protesters marching against Israeli actions. \\
        Hypothesis: People are have a protesting problems in Israeli. \\
        Label: entailment
    \end{quote}

    And there seems to be a common association between female identities and gossiping.

    \begin{quote}
        Premise: Two women sit outside on a bench along a street, looking across the street. \\
        Hypothesis: Two women sit outside on a bench along a street, looking across the street as they gossip with each other. \\
        Label: neutral
    \end{quote}

    \begin{quote}
        Premise: Two men in orange vests work in road construction. \\
        Hypothesis: Three women gossip in the break room. \\
        Label: contradiction
    \end{quote}

    \begin{quote}
        Premise: Three girls wear straw hats and dance on a stage. \\
        Hypothesis: 3 girls are painting their nails and gossiping in the locker room. \\
        Label: contradiction \\
    \end{quote}

    \begin{quote}
        Premise: Four girls are sitting in a hot tub. \\
        Hypothesis: The girls are gossiping. \\
        Label: neutral
    \end{quote}

    \subsection{Crowdsourcing Setup}

    As we have seen in previous sections, the premises sometimes contain biases.
    Meanwhile, they can also indirectly mislead crowd-workers into adding their own biases to the hypotheses by
    containing concepts, which are harm-free by themselves, that resonate with crowd-workers biases.

    A common characteristic of such cases is that they are mostly unrelated to the premises except for the main
    subject.
    This phenomenon is particularly obvious in \texttt{neutral} data samples, in which the crowd-workers have the most
    freedom for creative imagination.

    We believe with adequate instructions, a free-form generated data collection approach can be mostly free from
    biases.
    For example, specific warnings should be given to crowd-workers so that they do not imagine completely unrelated
    scenarios when writing hypotheses.
    In addition, they should be careful about introducing gender, race, nationality, and other identities that are
    unseen in the premises.
    Perhaps asking them for a short paragraph to justify their reasoning in such cases can discourage them from doing
    that.
    Meanwhile, this extra data can be used for building reasoning models.


    \section{Conclusion}

    In this report, we have analyzed the biases in SNLI dataset both quantitatively and qualitatively.
    In addition to biases in premises, we found much more in the hypotheses.
    Sometimes such biases are indirectly generated based on concepts in the premises, which encourages us to develop
    a good crowdsourcing paradigm that discourages crowd-workers from generating biases.

    \clearpage
    \bibliography{anthology,custom}
    \bibliographystyle{acl_natbib}

    % \appendix

    % \section{Example Appendix}
    % \label{sec:appendix}

    % This is an appendix.

\end{document}
